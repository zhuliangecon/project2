\documentclass[12pt,a4paper]{article}

% load packages
\usepackage{xcolor}
\usepackage{graphicx}
\usepackage{amsmath}
\usepackage{amsfonts}
\usepackage{amssymb}
\usepackage{listings}
\usepackage{hyperref}
\usepackage{algorithm}
\usepackage{algpseudocode}
\usepackage[left=2cm,right=2cm,top=2cm,bottom=2cm]{geometry}

% set code style
\definecolor{codegreen}{rgb}{0,0.6,0}
\definecolor{codegray}{rgb}{0.5,0.5,0.5}
\definecolor{codepurple}{rgb}{0.58,0,0.82}
\definecolor{backcolour}{rgb}{0.95,0.95,0.92}

% set lst style
\lstdefinestyle{mystyle}{
    backgroundcolor=\color{backcolour},   % 设置背景颜色
    commentstyle=\color{codegreen},
    keywordstyle=\color{magenta},
    numberstyle=\tiny\color{codegray},
    stringstyle=\color{codepurple},
    basicstyle=\ttfamily\footnotesize,    % 使用等宽字体
    breakatwhitespace=false,         
    breaklines=true,                 
    captionpos=b,                    
    keepspaces=true,                 
    numbers=left,                    
    numbersep=5pt,                  
    showspaces=false,                
    showstringspaces=false,
    showtabs=false,                  
    tabsize=2
}
\lstset{style=mystyle}


\title{Project 1 Report for Probelm 1.1}
\author{Zhu Liang}
\date{\today}

\begin{document}

\maketitle

\section{Project Description}
% Describe the problem you are trying to solve.
The objective of this project is to write a parallel program in C that prints ``Hello from $<$rank$>$ of $<$size$>$ processors." in order by processors' rank. 
Test it for a system with at least 13 processors.

\section{Algorithm Description}
This program utilizes the MPI parallel programming library to execute the code in parallel.
The primary objective of the program is to have each processor print its identifier in order. 
To ensure the printing is in order, the program employs the \texttt{MPI\_Barrier} function to synchronize all processes.
Then, each process will parallelly run the same for-loop and only print its identifier when its rank matches the loop index. 
After printing, the program will synchronize all processes again to ensure the printing is in order.


The pseudo-code of the program \texttt{hello.c} is shown in Algorithm 1. 
This program is based on the coding from lecture slides \cite{deng2023principles} and compiled by the shell script \texttt{project1.sh}.

\begin{algorithm}
    \caption{Fix Hello Order}
    \begin{algorithmic}[1]
    \Procedure{FixHelloOrder}{}
    \State \textbf{Initialize} MPI environment
    \State \textbf{Get} rank and size of processes
    \State \textbf{Synchronize} all processes
    \For{each process}
        \If{current process rank matches loop index}
            \State \textbf{Print} ``Hello from $<$current rank$>$ of $<$size$>$ processors."
        \EndIf
        \State \textbf{Synchronize} all processes
    \EndFor
    \State \textbf{Finalize} MPI environment
    \EndProcedure
    \end{algorithmic}
\end{algorithm}

\section{Results}
The program was executed on a \texttt{short-28cores} node by the shell script \texttt{project1.sh}. The output of the program is shown below.
\begin{lstlisting}[language=bash]
    Hello from 0 of 14 processors.
    Hello from 1 of 14 processors.
    Hello from 2 of 14 processors.
    Hello from 3 of 14 processors.
    Hello from 4 of 14 processors.
    Hello from 5 of 14 processors.
    Hello from 6 of 14 processors.
    Hello from 7 of 14 processors.
    Hello from 8 of 14 processors.
    Hello from 9 of 14 processors.
    Hello from 10 of 14 processors.
    Hello from 11 of 14 processors.
    Hello from 12 of 14 processors.
    Hello from 13 of 14 processors.
    
    real	0m0.655s
    user	0m0.476s
    sys	0m5.228s
    
    
\end{lstlisting}


\section{Analysis}
The program successfully prints the identifier of each process in order. 
The program can be executed on a system with at least 13 processors. 

% mention the general idea of the algorithm and why it reveals the difference  of parallel and serial
The program run the same for-loop on each process and synchronize the processes after each iteration. 
It ensures that each process will print its identifier in order.
The program is parallel because each process runs the same for-loop in parallel.
Although it is not the most efficient in term of performance, it is a simple way to solve the problem.

\section*{File Notes}
% mention folder name: project1; source code file name: hello.c; shell script file name: project1.sh
% mention how to compile and run the program

The source code of the program is in the \texttt{project1} folder. The source code file is named \texttt{hello.c}. The shell script file is named \texttt{project1.sh}. To compile and run the program, run the shell script \texttt{project1.sh} in Seawulf. The output will be printed on \texttt{output.txt}

\begin{thebibliography}{1}

    \bibitem{deng2023principles}
    Yuefan Deng.
    \textit{Principles of Parallel Computing: High-performance Computing and Algorithms for Big Data, Topic 3 Software \& MPI}.
    2023, p.30.
    
\end{thebibliography}


\end{document}
